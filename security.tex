\section{Security}

This hybrid solution allows us to use the parent chain as a
reliable protector against most of attacks that target the PoS systems.
\cite{pos_attacks}
In this section we assume that the parent chain is well secured and every
user has a decent and reliable access to it.

\subsection{Nothing at stake}

This attack splits into two cases: microforks and generational forks.

\subsubsection{Microforks}

When a malicious leader is elected they can produce blocks without any cost.
They are free to create conflicting branches within a single generation.

This case is very similar to BitcoinNG's. We introduce the Proof of Fraud (PoF)
mechanism to punish the malicious leaders, but in this situation we are able to
make the penalties much more severe by acting not only on the transaction fees,
but also staked tokens. On the other hand this kind of forking is
not dangerous at all – it may introduce some mess but will be instantly solved
with the next keyblock.

$$insert\ some\ image$$

\subsubsection{Generational forks}

This kind of attack is not a problem in PoW based BitcoinNG as producing
keyblocks is a hard task. On the contrary keyblocks on hyperchains are really,
really cheap - a malicious leader might flood the network with conflicting
keyblocks:

$$insert\ some\ image\ (aciiart\ in\ the\ latex\ code)$$
% K2a   K2b    K2c   K2d
% |     |      |     |
% V     V      V     V
% K1 <-- O <-- O < -- O <-- O 

The $K2a$, $K2b$, etc. are keyblocks on the child chain emitted by a malicious
leader over different microblocks (denoted by the $O$ letter). This
effectively splits the network in many parts, as it becomes unclear to which of
the forks the delegates should commit. To resolve this issue a single leader
must be agreed upon using some additional mechanism – a new election
should be performed with exclusion of the compromised delegate.

In order to notify the network about the generational fork the peers need to announce
this fact on the parent chain by publishing the fraud commitments along with the
cryptographical proofs of generational fraud (PoGF). Each commitment has to point to
the latest child keyblock considered by the delegates to be
valid, that is the generation where the fork starts. The commiter
needs also to submit to which fork they want to contribute – if they get elected they
will must keep building on it (we disallow rollbacks). The voting power should be
calculated basing on the latest block before the fraud was detected.

Due to network propagation
delays and connectivity failures some nodes might not notice that a generational
fork was created and not respond accordingly. New nodes or the ones catching up
after downtime need to be aware that a generational fork was created and apply
the appropriate resolution. Some of them may commit to one of the forks and receive a fraud
notification after they started mining. Therefore we introduce a metric over the
branches which for conventional reason we are going to call "difficulty". The
rule to resolve conflicts caused by such dataraces is similar to already
existing PoW strategy "follow the most difficult chain", and the formula goes
as follows:
\begin{minipage}{\linewidth}
\begin{lstlisting}
  difficulty : Block -> Int
  difficulty(Genesis) = 0
  difficulty(block) =
    if exists proof of generational fraud on /block/
    then sum of the voting power of delegates
           that commited to /prev(block)/ +
         sum of the voting power of delegates
           that commited to solve the genfork
    else sum of the voting power of delegates
           that commited to /prev(block)/
\end{lstlisting}
\end{minipage}

If two forks have the same difficulty then we prefer the one with lower
blockhash. This formula ensures that keyblocks pointing to a PoGF are always
strongly preferred over ordinary keyblocks – delegates who detect generational
forks have higher priority over poorly connected/bootstraping/syncing ones.

This solution may look vunerable to situations where the leader doesn't respond
or responds with significant delay. To solve these cases we just elect new
leader stalling the network for a while. To vanish the doubts araising when the
previous leader reapperas we can assume
finalization after $f$ (implementation dependent) generations.

There is an interesting case where the malicious leader submits a generational
fork by publishing keyblocks on conflicting microblocks.
Here we want to prioritize PoGF because the consequences of the forks on
keyblocks are much more severe than these on microblocks and we would need to
solve them anyway.

\subsection{Stake grinding}

Since the RNG depends ultimately on the keyblock hash on the PoW chain it is
impossible to predict its outcomes. One could try to mine the parent chain
in a special way, but it would require so much computational power that in
most cases it would be easier just to take the control over it by some 51\%
attack.

\subsection{Long range attack}
While it is still possible to perform it, it would be impossible to do it in
silence and without preparation since the very beginning. The commitments
guarantee that the information of the delegates is stored on a immutable chain
and one would need to announce their will of mining some suspicious blocks during
the full period of the attack. This would easily expose the will of the attacker
and would let the others prepare for eventual surprise (by for example
blacklisting them).

\subsection{Avoiding punishments}
Depending on the circumstances on the network the transaction fees may vary
– this makes the original penalty system from BitcoinNG not sufficient in every
case as the risk of fraud detection may be much lesser than expected profit.

One of the most natural ideas is to freeze the stake with some period before the
election. In this scenario the protocol will be able to painfully slash the
malicious leaders by burning/redistributing their stake. This however may raise
some problems when the delegated voting is used – the malicious leader may vote
for their second, empty account that will do the fraud loosing potentially
nothing in case of compromitation. This scenario can be dealt with allowing only
top $k$ stakers to be voted on or slashing \textit{everyone} who supported the
malicious leader. We leave this implementation dependent as different solutions
require different security approaches.