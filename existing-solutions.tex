\section{Existing solutions}

In this section, we describe the existing approaches to the problem along with
the problems they face and how they attempt to address them.

\subsection{Proof of Work}

Proof of Work (PoW) addresses the problem of decision making by forcing the
users (here called miners) to solve some hard computational puzzle to validate
(here, mine) blocks\cite{bitcoin}. The point is to make it hard to dominate the
network by a single selfish entity. This approach works as long as nobody holds
over 50\% of the whole computational power, in which case they could fork the
chain at any point and get ahead of the main history line. This is a serious
issue since in most protocols the most difficult chain is considered the proper
one. Therefore one needs a lot of participants in the network to make it
reasonably safe. Moreover, this approach leads to extreme waste of energy and
huge costs---according to some measurements, the whole blockchain environment
burns enough energy to power entire Denmark\cite{bitcoin_energy}.

This idea does not scale well---it is almost impossible to create a public
network from scratch that would not eventually be dominated by some malicious
entity. A lot of existing serious blockchains suffer this
problem\cite{51attack}. On the other hand, a network becomes extremely secure
once it is popular enough.

\subsection{Proof of Stake}

While PoW distributes the leadership based on computational power, PoS does it
based on so-called stake, which in most cases means token supply, sometimes with
additional tweaks\cite{peercoin}\cite{cryptocurr_without_pow}. The idea is to
create a leadership voting system which is activated periodically. Each time an
election event occurs, the new leader is randomly selected from the stakeholders
(called delegates). The chance of winning an election is proportional to the
size of one's stake. This approach does not incur any noticeable energetic
overhead and therefore is much more friendly to the environment. It also does
not require users to have powerful computers to be able to have some involvement
in decision making.

However, PoS comes with some serious issues. First of all, there is the infamous
``nothing at stake''\cite{pos_flaws_nothing} problem, which exploits the lack of
any cost of the actual mining. In this case, there is no downside to staking
several branches simultaneously in case of a fork. Some protocols address it by
introducing a ``slashing'' mechanism to burn stake of the
attacker\cite{eth_slashing}. However, that does not prevent the attack itself,
but only decreases incentives to do it.

Ensuring that the source of entropy is distributed along the chain, makes all of
the elections entirely deterministic and predictable leading to a strategy known
commonly as ``stake grinding,'' where the dishonest leader tries to rearrange the
transactions to influence the result of the upcoming election.

Next issue is the ``long-range attack''\cite{pos_flaws_long}. In the very
beginning, the stake is scattered among a small group of delegates that together
have full control over the chain. After some time, they can cooperate and start
a concurrent chain diverging from the main one. This could lead to nasty frauds
and would destabilize trust over the chain.

On the other hand, there are multiple approaches to deal with these problems.
For instance, to prevent nothing-at-stake, the CASPER protocol introduces a
``wrong voting penalty,'', which punishes voters who support conflicting
forks\cite{casper}. However, this solution is backed by a finality gadget, which
is located on another blockchain anyway. NXT deals with long-range attack by
forcefully finalizing all blocks that are older than 720 generations\cite{nxt}.
While this approach is stable for well established networks with high uptime, it
introduces weak subjectivity since one still needs to trust some entity while
entering the network for the first time or after a longer downtime. Ouroboros
staking system has managed to reach solid security, but at the cost of very high
complexity\cite{ouroboros}.

Although these are only few examples, the overall argument is that classical PoS
comes with many problems, which, as they are being solved, eventually introduce
new ones. This undermines reliability of many PoS systems, especially compared
to mature PoWs.
