
\section{Existing solutions}

In this section we will describe the existing approaches to the problem along
with the problems they suffer and how are they trying to face them.

\subsection{Proof of Work}

Proof of work (or shortly PoW) solves the problem of randomness by forcing the
users (here called miners) to solve some hard computational puzzle to validate
(here, mine) blocks \footnote{https://bitcoin.org/bitcoin.pdf}.
The idea is to make it hard to dominate the network by a
single entity. This solution works as long as nobody has over 50\% of the whole
computational power – in that case they could just fork the chain at any point
and get ahead of the main history line (this is a serious issue since in most
protocols the longest chain is considered the actual one). Therefore you need a
lot of participants in the network to make it reasonably safe. Moreover, this
solution leads to extreme waste of energy and huge costs – according to some
measurements the whole blockchain environment wastes enough energy to power
Denkmark
\footnote{https://arstechnica.com/tech-policy/2017/12/bitcoins-insane-energy-consumption-explained/}.

This idea does not scale well – it is almost impossible to create a public
network from scratch that wouldn't be eventually dominated by some malicious
user. No mater if for some particular benefits or just for fun. Actually, a lot
of existing serious blockchains suffer this problem
\footnote{https://www.crypto51.app/}. On the other hand it is extremely solid if
popular enough.

\subsection{Proof of Stake}

While PoW distributes the leadership by computational power, the PoS does it by
the so-called stake, which basicaly means the token supply (sometimes decorated
with some additional tweaks)
\footnote{https://www.peercoin.net/whitepapers/peercoin-paper.pdf}
\footnote{https://link.springer.com/chapter/10.1007/978-3-662-53357-4_10}.
The idea is to create a leadership voting system
that is triggered with some period of time. Each time the election event occurs
the new leader is randomly selected among the stakeholders (called delegates).
The election chance is proportional to the size of the stake. This approach does
not require any burning of empty CPU cycles and therefore is way better for the
environment. It doesn't require users to have powerful computers to be able to
have some involvement in the decision making.

However, it comes with some serious issues. First comes the "nothing at stake"
\footnote{https://medium.com/@abhisharm/understanding-proof-of-stake-through-its-flaws-part-2-nothing-s-at-stake-8d12d826956c}
problem which exploits the lack of any cost of the actual mining. In this case
there is no reason not to mine several branches simultanousely in case of fork.
The next thing is the source of random entropy – as it needs to be deterministic
and distributed it usually comes from the last blocks' hashes. However this
makes all of the elections completely predictible and leads to strategy known
commonly as "stake grinding", where the dishonest leader tries to rearrange the
transactions to influence the result of the upcoming election. Next issue is the
"long range attack".
\footnote{https://medium.com/@abhisharm/understanding-proof-of-stake-through-its-flaws-part-3-long-range-attacks-672a3d413501}
In the very beginning the stake is scattered among a small
group of delegates that toghether have full control over the chain. After
several years they can agree and start a concurrent chain from the same genesis
and go ahead of the main one. This could lead to nasty frauds and would
destabilize the trust over the chain.

However, there are multiple approaches to deal with these problems. For instance
the CASPER protocol introduces a "wrong voting penalty" which punishes the
voters that support conflicting forks to deal with nothing at stake
\footnote{https://arxiv.org/pdf/1710.09437.pdf}.
However, there is PoW mechanism in the background anyway. Another way is to use
slashing mechanisms known from BitcoinNG
\footnote{https://www.usenix.org/system/files/conference/nsdi16/nsdi16-paper-eyal.pdf},
but they are hopeless in a situation where the chain forks over a transation
transferring stake between two accounts. Also, it simply doesn't work until the
conflicting branch is published. NXT deals with long range attack by forcefully
finalizing all blocks that are older than 720 generations
\footnote{https://nxtdocs.jelurida.com/Nxt_Whitepaper}.
One must note that this doesn't actually solve the problem, but rather moves it
away. Moreover, it introduces weak subjectivity since one still needs to trust
some entity while entering the network for the first time.

These are only some examples. The general situation is that PoS comes with a lot
of flaws that are being solved by eventually introducing some other ones. This
makes most of the pure PoS networks not reliable, and especially not as reliable
as mature PoWs.