\section{Existing solutions}

In this section, we will describe the existing approaches to the problem along
with the issues they suffer and how they are trying to face them.

\subsection{Proof of Work}

Proof of Work (or shortly PoW) solves the problem of decision making by forcing the
users (here called miners) to solve some hard computational puzzle to validate
(here, mine) blocks\cite{bitcoin}.
The point is to make it hard to dominate the network by a
single selfish entity. This solution works as long as nobody has over 50\% of the whole
computational power – in that case they could just fork the chain at any point
and get ahead of the main history line (this is a serious issue since in most
protocols the longest chain is considered the proper one). Therefore one needs a
lot of participants in the network to make it reasonably safe. Moreover, this
solution leads to extreme waste of energy and huge costs – according to some
measurements, the whole blockchain environment burns enough energy to power
the whole Denmark\cite{bitcoin_energy}.

This idea does not scale well – it is almost impossible to create a public
network from scratch that wouldn't be eventually dominated by some malicious
entity. A lot of existing serious blockchains suffer this problem
\cite{51attack}. On the other hand, it is extremely solid if
the network is popular enough.

\subsection{Proof of Stake}

While PoW distributes the leadership by computational power, the PoS does it by
the so-called stake, which in most cases means the token supply (sometimes decorated
with some additional tweaks)\cite{peercoin}\cite{cryptocurr_without_pow}.
The idea is to create a leadership voting system
that is triggered by some amount of time. Each time the election event occurs,
the new leader is randomly selected among the stakeholders (called delegates).
The election chance is proportional to the size of the stake. This approach does
not imply any noticeable energetic overhead and therefore is much more friendly for the
environment. It also doesn't require users to have powerful computers to be able to
have some involvement in decision making.

However, it comes with some serious issues. First of all, there is the infamous "nothing at stake"
\cite{pos_flaws_nothing}
problem, which exploits the lack of any cost of the actual mining. In this case,
there is no downside to staking several branches simultaneously in case of a fork.

Ensuring that the source of entropy is distributed along the chain itself,
makes all of the elections entirely deterministic and predictable leading to a strategy
known commonly as "stake grinding," where the dishonest leader tries to rearrange the
transactions to influence the result of the upcoming election.

Next issue is the "long range attack"\cite{pos_flaws_long}.
In the very beginning, the stake is scattered among a small
group of delegates that together have full control over the chain. After
some time, they can agree and start a concurrent chain from the same genesis
and go ahead of the main one. This could lead to nasty frauds and would
destabilize the trust over the chain.

On the other hand, there are multiple approaches to deal with these problems. For instance,
the CASPER protocol introduces a "wrong voting penalty," which punishes the
voters that support conflicting forks to deal with nothing at stake\cite{casper}.
However, there is a PoW mechanism in the background anyway. Another way is to use
proof of fraud mechanisms known from BitcoinNG\cite{bcng},
but they are hopeless in a situation where the chain forks over a transaction
transferring stake between two accounts, and the penalties are not always
sufficient. Also, it simply doesn't work until the
conflicting branch is published. NXT deals with long-range attack by forcefully
finalizing all blocks that are older than 720 generations\cite{nxt}.
One must note that this doesn't actually solve the problem, but rather moves it
away. Moreover, it introduces weak subjectivity since one still needs to trust
some entity while entering the network for the first time or after a longer downtime.

These are only some examples. The general point is that PoS comes with many flaws
that are being solved by eventually introducing some other ones. This
makes most of the pure PoS networks not reliable, and especially not as reliable
as mature PoWs.
