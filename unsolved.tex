\section{Derived issues}

While the presented idea seems to cover the majority of cases, some scenarios
may break the system's stability. However, depending on the chosen parent chain
and the tweaks of inner protocol, their impact can be limited to an acceptable
risk.

\subsection{Forks of the parent}

Whatever happens to the parent, the similar shall happen to the child. The child
chain is entirely vulnerable to forks of the parent chain, and it is quite hard
to agree about on which branch to continue. Most likely, the hyperchain would
fork as well. On the other hand, if the validators manage to decide on one
branch, it would be technically possible to jump into the other if the chosen
one becomes less attractive.

\subsection{Attacks on the parent}

The hyperchain can never be more secure than the parent. If somebody succeeds in
a 51\% attack on the parent chain, they will also take control of the child
chain. Therefore the choice of the parent should be made with regards to its
security.

\subsection{Finalization time}

Since there is no single correct strategy on how to react to forks, the
finalization time shall not be shorter on a hyperchain than on the parent chain.
If the parent key block gets rolled back, so will all of the leader elections.

\subsection{Stake collusion}

While stake distribution may look healthy on the surface, stakers can actually
collude rather than act as self-serving actors. As a consequence, there may
emerge a group of delegates whose perceived best interest would be to join a
cartel and unfairly influence the blockchain. This issue, however, is common to
all capital-based PoS systems where power is handed by the value staked in.
Although we do not propose any direct solution to this, a properly tailored
staking contract may address this issue.
