\section{Derived issues}

While the presented idea seems to cover most of the cases, there are some
scenarios that may break the stability of the system. However, depending on the
chosen parent chain and the tweaks of the inner protocol their impact can be
limited to some acceptable risk.

\subsection{Forks of the parent}
Whatever happens to the parent, the similar shall happen to the child. The child
chain is completely vunerable to forks of the parent chain and it is quite hard
to get an agreement on which branch should we continue. Most likely the
hyperchain would fork as well. On the other hand if the validators manage to
decide on one way, it would be technically possible to jump into the other
branch if the chosen one turns out to be less attractive.

\subsection{Attacks on the parent}

The hyperchain shall never be more secure than the parent. If somebody will
succeed in for example a 51\% attack on the parent chain, they will also take
the control of the child chain. Therefore the choice of the parent should be
done with regards to the security of it.

\subsection{Finalization time}

Since as mentioned there is no right strategy on how to react on forks, the
finalization time shall not be shorter on hyperchain than on the parent chain.
If the parent keyblock gets rolled back, so will all of the leader elections do.