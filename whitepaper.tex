%\documentclass[a4paper, 10pt, conference]{ieeeconf}      % Use this line for
%\input{settings}

\documentclass{article}

%\overrideIEEEmargins

\newcommand{\blockchain}{{\ae}ternity blockchain}
\newcommand{\aet}{{\ae}ternity}
\newcommand{\Aet}{{\AE}ternity}

\title{\huge \Aet\ Hyperchains \\[0.5em]
  \large Recycling power of blockchains
  \\[1em] v0.1.0-DRAFT }

\author{ Grzegorz Uriasz
  \and Radosław Rowicki
  \and Dimitar Ivanov
  \and Ulf Wiger
}
% See the \addtolength command later in the file to balance the column lengths
% on the last page of the document

%\\[0.4em]
%\blockchain\ block 224206 hash kh_28pjh8P7mbs2udZ2ai5WdERfLNWmupZW14K8umVVddsafRjwj6

%\usepackage{fontspec}
%\setmainfont{MinionPro-Regular.otf}

%\usepackage[]{todonotes} % notes not showed
\usepackage[draft]{todonotes}   % notes showed

\usepackage{hyperref}
\usepackage{graphics}
\usepackage{multicol}
\usepackage{csquotes}

% \BibTeX command to typeset BibTeX logo in the docs
\AtBeginDocument{%
  \providecommand\BibTeX{{%
    \normalfont B\kern-0.5em{\scshape i\kern-0.25em b}\kern-0.8em\TeX}}}

% See the \addtolength command later in the file to balance the column lengths
% on the last page of the document

%\addbibresource{references.bib}

\begin{document}
\maketitle


%
% The abstract is a short summary of the work to be presented in the article.
\begin{abstract}

The blockchain relies on consensus between miners. A well known problem is that the block producer should not be known in advance as this opens the possibilities for malicious actors. A solution as old as Satoshi Nakamoto’s whitepaper is that this must be a random event. This should also be an open, transparent and provable process but also a fair one. There are different approaches for achieving this and they all have their perks and flaws. We propose a hybrid between a Proof of Work and Proof of Stake.

\end{abstract}

%%\keywords{blockchain, state channels}

\tableofcontents

\newpage


\section{Introduction}

\textbf{TO BE REBUILT}

Proof of Work (PoW) solutions are traditionally the most popular solution. They achieve network security by burning (usually empty) CPU cycles. Assumption is that no single entity has over 51\% of the computing power of the network. This is a slow and costly process that relies on having a vast and decentralized network of miners.

Proof of Stake (PoS) on the other hand is much more energy efficient but its implementation details matter greatly. It could be subject to nothing-at-stake problem or stake grinding and thus eventually degrade to a PoW solution.

We propose a hybrid approach between the two: we build a PoS system that relies on a PoW solution for providing the security. The security of the PoW network itself is outside of the scope of this document: this is up to the specific hyperchain setup to choose a secure PoW network to use as it can ever be as secure as the PoW chain. It is worth noting that it should be possible to change the PoW system the Hyperchain is using further down in its lifetime. We will call the PoW chain a parent chain and the PoS - a child chain.

This approach is agnostic with regards of the structure of the blocks that makes up the Hyperchain. It had been designed with BitcoinNG in mind but it would work with a traditional chain of blocks as well. After some adaptations it could work with any serializable structure of blocks. The important architectural constraint is having just one block producer at a time.

The role of the block producer is a temporary one. The block producer is the one that includes new transactions in the blockchain. The block producer is elected, and once a new one is elected, the old one can no longer include transactions. A subset of of all accounts in the child chain is eligible for being elected as the next block producer. We call them delegates.

Their child chain balances represent their staking power in the child chain. Every account can use their staking power to become a delegate or to delegate it to another account. How staking power relates to becoming a delegate is up to the child chain.

Delegates are provable only in the scope of the child chain. The parent chain has no knowledge for the child chain's mechanism for who is a delegate.

A requirement for the parent chain is that anyone can post a transaction to it. Delegates use that to provide a certain hash. It represents the delegate’s intention to become a block producer and we call it a commitment. An example for such hash would be the block the delegate considers to be the last one one had seen. This would be the block one would append their block to. A delegate is free to post as many commitments they like but at most one would be considered as a valid one - the one with the expected hash of the previous block.

Once a certain event happens on the parent chain - the child chain election mechanism takes over. Using all valid commitments and a source of entropy, a new child chain block producer is elected deterministically and transparently by a publicly provable function. The source of entropy could be anything that is part of the parent chain (ex. a block hash). This function takes into account the staking power of every commitment’s delegate - the more staking power, the higher the chance of being elected. Delegate’s staking power is taken at commitment’s height.

After being elected as a block producer, one is expected to produce one or more blocks in the child chain. Once other delegates receive those - they start posting their new commitments on the parent chain and the election process begins all over again. If the elected block producer does not produce block(s) because of being missing or actively malicious, delegates post their old commitments again.

This binds the child chain election process to the parent chain and the child chain reuses the safety of the parent chain. This allows making smaller and possibly private child chains that still keep the same amount of security the parent chain provides publicly. It is not for free as delegates are expected to post their commitments on the parent chain. Assumption is that they’re to be propery reimbursed for those in the child chain either by the fees there or by freshly minted child chain coins.

This approach is intended to take advantage of the efficient PoS approach while keeping the safety of the PoW. To the best of our knowledge - it is a unique solution that provides this safety even for small Hyperchains. That perk makes it suitable for various private custom solutions but also for scaling the blockchain throughput.


\section{Existing solutions}

In this section we will describe the existing approaches to the problem along
with the problems they suffer and how are they trying to face them.

\subsection{Proof of Work}

Proof of work (or shortly PoW) solves the problem of decision making by forcing the
users (here called miners) to solve some hard computational puzzle to validate
(here, mine) blocks \footnote{https://bitcoin.org/bitcoin.pdf}.
The point is to make it hard to dominate the network by a
single selfish entity. This solution works as long as nobody has over 50\% of the whole
computational power – in that case they could just fork the chain at any point
and get ahead of the main history line (this is a serious issue since in most
protocols the longest chain is considered the proper one). Therefore one needs a
lot of participants in the network to make it reasonably safe. Moreover, this
solution leads to extreme waste of energy and huge costs – according to some
measurements the whole blockchain environment wastes enough energy to power
whole Denmark
\footnote{https://arstechnica.com/tech-policy/2017/12/bitcoins-insane-energy-consumption-explained/}.

This idea does not scale well – it is almost impossible to create a public
network from scratch that wouldn't be eventually dominated by some malicious
entity. A lot of existing serious blockchains suffer this problem
\footnote{https://www.crypto51.app/}. On the other hand it is extremely solid if
the network is popular enough.

\subsection{Proof of Stake}

While PoW distributes the leadership by computational power, the PoS does it by
the so-called stake, which in most cases means the token supply (sometimes decorated
with some additional tweaks)
\footnote{https://www.peercoin.net/whitepapers/peercoin-paper.pdf}
\footnote{https://link.springer.com/chapter/10.1007/978-3-662-53357-4\_10}.
The idea is to create a leadership voting system
that is triggered with some period of time. Each time the election event occurs
the new leader is randomly selected among the stakeholders (called delegates).
The election chance is proportional to the size of the stake. This approach does
not imply any noticable energetic overhead and therefore is much more friendly for the
environment. It also doesn't require users to have powerful computers to be able to
have some involvement in the decision making.

However, it comes with some serious issues. First of all, there is the infamous "nothing at stake"
\footnote{https://medium.com/@abhisharm/understanding-proof-of-stake-through-its-flaws-part-2-nothing-s-at-stake-8d12d826956c}
problem which exploits the lack of any cost of the actual mining. In this case
there is no downside to staking several branches simultaneously in case of a fork.
The next thing is the source of random entropy – as it needs to be deterministic
and distributed it usually comes from the chain itself. This
makes all of the elections completely predictible and leads to a strategy known
commonly as "stake grinding", where the dishonest leader tries to rearrange the
transactions to influence the result of the upcoming election. Next issue is the
"long range attack".
\footnote{https://medium.com/@abhisharm/understanding-proof-of-stake-through-its-flaws-part-3-long-range-attacks-672a3d413501}
In the very beginning the stake is scattered among a small
group of delegates that toghether have full control over the chain. After
some time they can agree and start a concurrent chain from the same genesis
and go ahead of the main one. This could lead to nasty frauds and would
destabilize the trust over the chain.

On the other hand, there are multiple approaches to deal with these problems. For instance
the CASPER protocol introduces a "wrong voting penalty" which punishes the
voters that support conflicting forks to deal with nothing at stake
\footnote{https://arxiv.org/pdf/1710.09437.pdf}.
However, there is a PoW mechanism in the background anyway. Another way is to use
slashing mechanisms known from BitcoinNG
\footnote{https://www.usenix.org/system/files/conference/nsdi16/nsdi16-paper-eyal.pdf},
but they are hopeless in a situation where the chain forks over a transaction
transferring stake between two accounts and the penalties are not always
sufficient. Also, it simply doesn't work until the
conflicting branch is published. NXT deals with long range attack by forcefully
finalizing all blocks that are older than 720 generations
\footnote{https://nxtdocs.jelurida.com/Nxt\_Whitepaper}.
One must note that this doesn't actually solve the problem, but rather moves it
away. Moreover, it introduces weak subjectivity since one still needs to trust
some entity while entering the network for the first time or after a longer downtime.

These are only some examples. The general point is that PoS comes with a lot
of flaws that are being solved by eventually introducing some other ones. This
makes most of the pure PoS networks not reliable, and especially not as reliable
as mature PoWs.

\section{Hyperchains Design}
\graphicspath{ {./images/} }

The previous approaches had a lot to offer, but considering the cons it is hard
to scale them reasonably. PoW seems to work well only with big
computational effort being burned and PoS suffers from a huge amount of security
holes that require very complicated algorithms that usually either don't solve
the problem at all or move it further to another layer of abstraction.

Here we present a hybrid strategy that will benefit from the stability of PoW
solutions but will offer the scalability of PoS systems. A Hyperchain is a
special kind of blockchain that sticks to an already existing chain. They are
going to be called child and parent chain respectively\cite{hyperchains}.

The parent chain can be almost any blockchain in the world. In general, we want
to use some big existing PoW based chains (at the time of writing, preferably
Bitcoin or Ethereum, but not limited to) to reuse their burned work to maintain
the stability of the child chain. We would also like to have
PoS-like election system to choose the leaders on the hyperchain. In this case,
however, we have a very reliable – and most important, unpredictable – source
of randomness – the state of the parent chain. The idea is not very new,
though – there is already some research made in this direction\cite{blockchain_random}.

Having this machinery, it seems natural to start a new election each time a
(key)block was mined on the parent chain. The next leader shall be chosen
depending on the hash of that block and selected with chances proportional to
their stake. The selection algorithm is abstract over this document – it is
up to the hyperchain to define the details.

\begin{figure}[h]
	\caption{Component Diagram}
	\centering
	\includegraphics[scale=0.5]{hyperchain_component}
\end{figure}

We define a group of leadership candidates called "delegates." Each delegate
needs to express their will in participation in the election for the upcoming
generation by publishing a commitment transaction onto the parent chain. It is
important to make it clear what is their point of view on the child chain and
over which block they are going to compete.
Therefore the commitment must consist of:
\begin{itemize}
\item The subject of delegation on the child chain
\item The block over the delegate is going to build
\item Signature of the delegate from the child chain
\end{itemize}

One of the important concepts of the commitment idea is to be able to rely on
the parent chain's stability. We want to treat it as a rigid skeleton
of the hyperchain, which can be achieved by proper block hash linking. The
elected leader will be required to publish a key block on the child chain with a
cryptographic proof (referencing the parent) of their right to lead the
upcoming generation and publish microblocks.

One dilemma that rises at this point is whether should the commitment reference
the latest keyblock or the microblock of the child chain. Referencing microblock
on the first sight looks more transparent, but we believe that it would
lead to massive forking (especially when some peers wouldn't receive all of the
blocks). The problem with referencing keyblock is that the next leader could
steal the transactions and post them in their microblocks. This, however, can be
faced with a smarter feeing strategy: instead of giving the full fee to the
miner, we can split it up and give the bigger part to the next leader that did
include the previous leader's microblocks in their continuation of the history,
as it happens in the BitcoinNG\cite{incentive_bcng}.
After getting elected, the new leader posts a keyblock on the child chain that
references the point on the parent, which proves their right to lead the
generation. Besides that, they need to reference the microblock from the previous
generation they want to mine on.


\begin{figure}[h]
	\caption{Hyperchains Design}
	\centering
	\includegraphics[scale=0.3]{hyperchains_design}
\end{figure}


\section{Security}

This hybrid solution allows us to use the parent chain as a
reliable protector against most of the attacks that target the PoS systems.
\cite{pos_attacks}
In this section, we assume that the parent chain is secured well, and every
user has easy and reliable access to it.

\subsection{Nothing at stake}

This type of attack splits into two cases: microforks and generational forks.

\subsubsection{Microforks}

A malicious leader may produce blocks without any cost.
They are free to create conflicting branches within a single generation.

This case is very similar to BitcoinNG's\cite{bcng}. We introduce the Proof of Fraud (PoF)
mechanism to punish malicious leaders, but in this situation, we can make
the penalties much more severe by acting not only on transaction fees
but also staked tokens. On the other hand, this kind of forking is not dangerous
at all. It may introduce some mess but becomes solved instantly with the next key block.

\begin{figure}[h]
	\caption{Microfork}
	\centering
	\includegraphics[scale=0.35]{microfork}
\end{figure}

\subsubsection{Generational forks}

This kind of attack is not a problem in PoW based BitcoinNG as producing
key blocks is a hard task. On the contrary key blocks on hyperchains are
extremely cheap — a malicious leader may flood the network with coinciding
key blocks:

\begin{figure}[h]
	\caption{Generational Fork}
	\centering
	\includegraphics[scale=0.35]{genfork}
\end{figure}

The $K1a$, $K1b$, etc. are key blocks on the child chain emitted by a malicious
leader over different micro blocks. This
effectively splits the network into many parts, as it becomes unclear to which of
the forks the delegates should commit. To resolve this issue, a single leader
must be agreed upon using some additional mechanism – a new election
should be performed with the exclusion of the compromised delegate.

To notify the network about the generational fork, the peers need to announce
this fact on the parent chain by publishing fraud commitments and
cryptographic proofs of generational fraud (PoGF). Each commitment has to point to
the latest child key block considered valid by delegates or,
 in other words, to the generation where the fork starts. The committer also needs
to declare to which fork they want to contribute – if they get elected, they
will be required to build on it (we disallow rollbacks). The voting power should be
calculated based on the latest block before the fraud was detected.


\begin{figure}[h]
	\caption{Generational Fork Solving}
	\centering
	\includegraphics[scale=0.45]{genfraud}
\end{figure}

Due to network propagation
delays and connectivity failures, some nodes might not notice that a generational
fork was created and not respond accordingly. New nodes or the ones catching up
after downtime, need recognize that a generational fork was created and apply
the appropriate solution. Some of them may commit to one of the forks and receive a fraud
notification after they start mining. Therefore, we introduce a metric over the
branches that we call "difficulty" for a conventional reason. The
rule to resolve conflicts caused by such data races is similar to the already
existing PoW strategy "follow the most difficult chain," and the formula is
as follows:

\begin{minipage}{\linewidth}
\begin{lstlisting}
  difficulty : Block -> Int
  difficulty(Genesis) = 0
  difficulty(block) =
    if(exists proof of generational fraud on /block/)
         sum of the voting power of delegates
           that committed to /prev(block)/ +
         sum of the voting power of delegates
           that committed to solve the genfork
    else sum of the voting power of delegates
           that committed to /prev(block)/
\end{lstlisting}
\end{minipage}

If two forks have the same difficulty, then the one with lower block hash
is selected. This formula ensures that key blocks pointing to a PoGF are always
strongly preferred over ordinary key blocks – delegates who detect generational
forks have higher priority over poorly connected/bootstrapping/syncing ones.

This solution is vulnerable to situations where the leader does not respond
or responds with significant delay. In these cases, a new election is held,
stalling the network for a while. To resolve the doubts arising when the
previous leader reappears,
finalization after $f$ (implementation–dependent) generations is introduced.

In particular case a malicious leader may submit a generational
fork by publishing key blocks on conflicting micro blocks.
In response we prioritize PoGF, because the consequences of forks on
key blocks are much more severe than those on micro blocks, and they would need to
be resolved anyway.

\subsection{Stake grinding}

Since the RNG depends ultimately on the key block hash on a PoW chain, it is
impossible to predict its outcomes. One could try to mine the parent chain
in a special way, but it would require so much computational power that in
most cases it would be easier to take control by 51\% attack.

\subsection{Long–range attack}
While it is still possible to perform a long–range attack, it would be impossible to do it
secretly and without preparation since the very beginning. The commitments
guarantee that the information of delegates is stored on an immutable chain,
and one would need to announce their will of mining suspicious blocks during
the entire period of the attack. This would quickly expose the intention of the attacker
and let the others prepare for a possible upcoming attack
(by blacklisting them, for example).

\subsection{Avoiding punishments}
Transaction fees may vary depending on circumstances on the network.
Consequently, the original penalty system of BitcoinNG is not sufficient in
those cases, where the risk of fraud detection is much lesser than expected profit.

One of the most natural ideas is to freeze the stake for some period before the
election. In this scenario, the protocol is able to painfully slash
malicious leaders by burning/redistributing their stake. However, this may raise
some problems when delegated voting is used – a malicious leader may vote
for their second, empty account that will commit the fraud, losing potentially
nothing if compromised. This can be dealt with, allowing only
top $k$ stakers to be voted on or slashing \textit{everyone} who supported the
malicious leader. We leave this issue implementation–dependent as different solutions
require different security approaches.

\section{Derived issues}

While the presented idea seems to cover the majority of cases, some
scenarios may break the system's stability. However, depending on the
chosen parent chain and the tweaks of inner protocol, their impact
can be limited to an acceptable risk.

\subsection{Forks of the parent}

Whatever happens to the parent, the similar shall happen to the child. The child
chain is entirely vulnerable to forks of the parent chain, and it is quite hard
to agree about on which branch to continue. Most likely, the
hyperchain would fork as well. On the other hand, if the validators manage to
decide on one branch, it would be technically possible to jump into the other
if the chosen one becomes less attractive.

\subsection{Attacks on the parent}

The hyperchain can never be more secure than the parent.
If somebody succeeds in a 51\% attack on the parent chain,
they will also take control of the child chain.
Therefore the choice of the parent should be made with regards to its security.

\subsection{Finalization time}

Since there is no single correct strategy on how to react to forks, the
finalization time shall not be shorter on a hyperchain than on the parent chain.
If the parent key block gets rolled back, so will all of the leader elections.



\bibliographystyle{acm} \bibliography{paper}

\end{document}
