\section{Hyperchains}

\textbf{todo: reorganize, mention commitments earlier, images!}

The previous approaches had a lot to offer, but considering they cons it is hard
to scale them in a reasonable way. PoW seems to work well only with big
computational effort being burned and PoS suffers from huge amount of security
holes that require very complicated algorithms that usually either don't solve
the problem at all or just move it further to another layer of abstraction.

Here we present a hybrid strategy which will benefit from stability of PoW
solutions but will offer the scalability of PoS systems. A Hyperchain is a
special kind of blockchain that sticks to an already existing chain. They are
going to be called respectively child- and parent-chain
\footnote{https://medium.com/@yanislav/hyperchains-secure-cheap-scalable-blockchain-technology-for-everyone-3ddec96a4152}.

The parent chain can be almost any blockchain in the world. In general, we want
to use some big existing PoW based chains (at the time of writing, preferably
BitCoin or Ethereum, but not limited to) to reuse their burned work to maintain
the stability of the childchain. We would also like to have
PoS-like election system to choose the leaders on the hyperchain. In this case
however we have a really reliable – and most important, unpredictable – source
of randomness – the keyblock hash of the parent chain. The idea is not very new
though – there is already some research made in this direction
\footnote{https://eprint.iacr.org/2015/1015.pdf}.
The critics say that it is still possible to exploit it to mine blocks in such a
manner that the outcome would be beneficial, but in reality that reduces to
compound PoW – most likely the difficulty of mining such a block will go
squared, so in order to have actual control over the entropy one needs to
control the parent chain anyway.

Having this machinery it seems natural to start a new election each time a
keyblock was mined on the parent chain. The next leader shall be chosen
depending on the hash of that block and selected with proportional chances to
their stake. The selection algorithm is going to be straightforward – we take
the hash (let's say, MD5) and consider a whole MD5 counterdomain as a closed
line segment divided into intervals of lengths proportional to the stakes of the
delegates. The intervals are going to be sorted by the order of the respective
commitments that appeared on the parent chain. The generated hash will then
point into some subsection which will determine the winner.

$$Insert\ some\ nice\ image\ here$$

One of the important concepts of the commitment idea is to be able to rely on
the parent chain's stability. Therefore we want to treat it as a rigid skeleton
of the hyperchain which can be achieved by proper blockhash linking. Each
commitment must declare over which block is the delegating going to compete.
Therefore the commitment must consist of:
\begin{itemize}
\item The subject of delegation on the child chain
\item The block over the delegate is going to build
\item Singature of the delegate from the child chain
\end{itemize}

One dillema that rises at this point is whether should the commitment reference
the latest keyblock or the microblock of the child chain. Referencing microblock
on the first sight looks more consistent, but we believe that in reality would
lead to massive forking (especially when some peers wouldn't receive all of the
blocks). The problem with referencing keyblock is that the next leader could
steal the transactions and post them in their microblocks. This however can be
faced with the smarter feeing strategy: instead of giving the full fee to the
miner we can split it up and give the bigger part to the next leader that did
include the previous leader's microblocks in their continuation of the history.

$$Insert\ some\ nice\ image\ here$$


