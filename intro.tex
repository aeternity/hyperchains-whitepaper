\section{Introduction}

One of the most important problems that blockchains need to tackle in order to
maintain a stable and safe infrastructure is the need for a decentralized
consensus among the participants. This usually reduces to existence of some
point of agreement from which everyone should be able to derive some objective
and consistent truth about the state of the abstracted universe. The most common
case is that this point defines a certain description of the history, or sets
the person responsible for dictating it for some period of time. There is a
global tendention on depending on some random, fair and unpredictable event that
allows to build up a trustless agreement on the current status of events.

Proof of Work (PoW) solutions are traditionally the most popular ones.
They achieve network security by burning (usually empty) CPU cycles in order to
randomly hand the power to the users depending on their computational effort.
The assumption is that no single entity has over 51\%
of the computing power of the network to affect the outcome for their own
benefits. This is a slow and costly process that relies on having
a vast and decentralized network of miners. Proof of Stake (PoS) on the other
hand is much more energy efficient but its implementation details matter
greatly. It could be subject to the nothing-at-stake problem or stake grinding and
thus eventually degrade to an inelegant PoW solution.

We propose a hybrid approach between the two: we build a PoS system that
relies on an existing PoW network for providing security. The security of the
PoW network itself is outside of the scope of this document: it is up to
the specific hyperchain setup to choose a secure PoW network to use as it
can never be as secure as the PoW chain. It is worth noting that it should
be possible to change the PoW system the Hyperchain is using further down
in its lifetime. We will call the PoW chain a parent chain and the PoS-like
system - a child chain.

