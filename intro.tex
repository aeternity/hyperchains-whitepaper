\section{Introduction}

One of the most important problems that blockchains need to tackle to maintain a
stable and safe infrastructure is the need for a decentralized consensus among
the participants. This usually reduces to the existence of a point of agreement,
from which everyone should be able to derive an objective and consistent truth
about the state of the system. Most commonly, such point defines a certain
description of the universe's history, or establishes the person responsible for
dictating it for some time. There is a global tendency for depending on some
random, fair and unpredictable event that allows for building up a trustless
agreement on the current status of events.

Proof of Work (PoW) solutions are traditionally the most popular. They achieve
network security by burning (usually empty) CPU cycles in order to randomly hand
the power to users depending on their computational effort. This is a slow and
costly process, which relies on having a vast and decentralized network of
miners. On the other hand, Proof of Stake (PoS) is much more energy-efficient,
but its implementation details matter greatly. It could be subject to the
nothing-at-stake problem or stake grinding and thus eventually degrade to an
inelegant PoW solution.

We propose a hybrid approach called hyperchains: PoS systems, which rely on
existing PoW networks for providing security. The security of the PoW network
itself is outside the scope of this document: it is up to the specific
hyperchain setup to choose a secure PoW network since it can never be as secure
as the PoW chain. It is worth noting that it ought to be possible to change the
PoW system used by the hyperchain further down in its lifetime. From now on, we
will refer to the PoW chain as a parent chain and the PoS-like system---as a
child chain.
