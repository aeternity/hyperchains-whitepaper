\section{Election process and staking mechanism}

In this section, rather than prescribe specific mechanisms and rules, we propose
solutions that would facilitate planning and implementation of the desired
algorithms. It is not up this document to specify the details.

The most convenient way to organize the election process is to create a smart
contract on the hyperchain, which would manage the stake and evaluate PoF
penalties. This contract shall be referenced in the protocol, and its interface
should be specified there. It is advisable that the contract be adjusted by
regular calls on the fly — this could prevent some protocol–level hard forking.
If the VM supports it, the contract could also forcefully alter the blockchain
state (e.g. by using some internal Merkle tree framework). We highly recommend
introducing consensus changes with a decent delay to ensure that it will not
break during a true hard fork.

We propose the following features of a staking contract:

\begin{itemize}
\item Leader election
\item Voting power calculation
\item Delegates calculation
\item Voting power delegation
\item Applying punishments
\item Withdrawing and depositing the stake
\item (optional) Controlled hard forking
\end{itemize}

The system must implement some mechanisms to prevent abuse of the exposed
interface and it must be resistant to low responsiveness of the generation
leader. Therefore we highly recommend that some necessary calls (like election
initiation or reward claiming) be applied automatically in each key block and be
protocol–restricted in order to prevent arbitrary calls to them. The contract
may be free of any PoF validation and contain only the algorithm of issuing a
punishment to the current leader.
