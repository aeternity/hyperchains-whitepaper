\section{Election process and staking mechanism}

We do not want to force any particular mechanisms and rules in this section, but
rather propose some solutions that would make it easier to plan and implement
the desired algorithms. This is not up to this document to specify the details.

The most convenient way to organize the election process is to create a smart
contract on the hyperchain that would manage the stake and evaluate PoF
penalties. This contract shall be referenced in the protocol, and its interface
should be settled there. It is eligible for it to be adjusted by regular calls on the fly
– this could save some protocol-level hard forking. If the VM supports it, the
contract could also forcefully alter the blockchain state (i.e., by using some
internal Merkle tree framework). We highly recommend introducing consensus
changes with a decent delay to ensure that it won't break during a true hard fork.

We propose the following features of a staking contract:
\begin{itemize}
\item Leader election
\item Voting power calculation
\item Delegates calculation
\item Voting power delegation
\item Applying punishments
\item Withdrawing and depositing the stake
\item (optional) Controlled hard forking
\end{itemize}

The system must implement some mechanisms to prevent abuse of the exposed
interface and it must be resistant to low responsiveness
of the generation leader. Therefore we highly recommend making some necessary
calls (like election initiation or reward claiming) be applied automatically
in each keyblock and be protocol–restricted in order to prevent
arbitrary calls to them. The contract may be free of any PoF validation and
contain only the algorithm of issuing a punishment to the current leader.
